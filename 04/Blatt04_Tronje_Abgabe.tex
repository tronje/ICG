\newcommand{\authorinfotitle}{Tronje Krabbe}
\newcommand{\authorinfo}{Tronje Krabbe}
\newcommand{\titleinfo}{ICG Blatt 04}
\newcommand{\qed}{\square}
\newcommand{\tilD}{($\widetilde{\mathrm{D}}$)}

\documentclass [a4paper,11pt]{article}
\usepackage[german,ngerman]{babel}
\usepackage[utf8]{inputenc}
\usepackage[T1]{fontenc}
\usepackage{lmodern}
\usepackage{amssymb}
\usepackage{mathtools}
\usepackage{amsmath}
\usepackage{enumerate}
%\usepackage{breqn}
\usepackage{fancyhdr}
\usepackage{multicol}
\usepackage{eurosym}

\usepackage[a4paper,left=2cm,width=13cm,right=3cm]{geometry}

% for dem plots
\usepackage{pgfplots}
%\pgfplotsset{compat=1.10}
\usepgfplotslibrary{fillbetween}
% ---

\author{\authorinfotitle}
\title{\titleinfo}
\date{\today}

\pagestyle{fancy}
\fancyhf{}
\fancyhead[R]{\authorinfo}
\fancyhead[L]{ICG Hausaufgaben}
\fancyfoot[C]{\thepage}

\begin{document}
\maketitle
    \begin{enumerate}
        % Aufgabe 1
        \item[\textbf{1.}]
            Es gilt: bei naiver Multiplikation zweier Matritzen der Dimensionen $n \cdot m$ und $m \cdot p$ entstehen genau
            $n \cdot m \cdot p$ Multiplikationen. Additionen entstehen $6$ mal bei $3 \cdot 1$ und $18$ mal bei $3 \cdot 3$ sowie
            $2$ mal bei $2 \cdot 1$ und $4$ mal bei $2 \cdot 2$ Multiplikationen.
        \begin{enumerate}
        \item[a)]
            \begin{enumerate}
                \begin{tabular}{l | c | c || c}
                    & Multiplikationen & Additionen & Operationen\\
                    \hline
                    (a1) & 9 & 6 & 15\\
                    (a2) & 36 & 24 & 60\\
                \end{tabular}
            \end{enumerate}
        \vspace{10pt}
        \item[b)]
                \begin{tabular}{l | c | c || c}
                    & Multiplikationen & Additionen & Operationen\\
                    \hline
                    (b1) & 4 & 2 & 6\\
                    (b2) & 0 & 2 & 2\\
                    (b3) & 12 & 6 & 18\\
                    (b4) & 4 & 4 & 8\\
                    (b5) & 4 & 4 & 8\\
                    (b6) & 0 & 4 & 4\\
                \end{tabular}
        \end{enumerate}

        Berechnungen mit inhomogenen Koordinaten benötigen weniger Operationen bzw. Laufzeit als Berechnungen
        mit homogenen Koordinaten.\\
        Dagegen haben homogene Koordinaten den Vorteil, dass Punkte in der Unendlichkeit trotzdem repräsentiert werden können.
        Außerdem sind Berechnungen mit homogenen Koordinaten meist simpler.


    \end{enumerate}
\end{document}
